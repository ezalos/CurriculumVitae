\experienceheader{Stage de Recherche en Machine Learning}{Natixis}{02/2020 - 08/2020}
\experiencedescription{La détection d'attaques furtives par analyse de signaux faibles

Aujourd'hui, la lutte contre les malwares dans les entreprises est principalement une défense en réponse. Dans un but de défense proactive, ce projet fut l'occasion d'étudier et de mettre en pratique les dernières avancées permises par l'IA.
À travers la méthodologie Agile, ce stage s'est déroulé lors de la première édition du lab 42 / Natixis.
}
\experiencesimplelist{ Realisations: }{\begin{tightitemize}
\item {Création d’un dataset labélisé :}
\item {Sélection d'événements intéressants en collaboration avec les analystes de sécurité.}
\item {Analyse statistique, puis nettoyage de la donnée labellisée.}
\item {Création de features :}
\item {État de l'art sur la création de features (Machine Learning / Sécurité informatique)}
\item {Création de features depuis des données brutes textuelles (logs Windows/Linux/Apps)}
\item {Création d'un parseur de Command-Line en REGEX, résistant contre les techniques d'offuscations.}
\item {Mise en place de tables de correspondances CSV}
\item {Création automatique de dictionnaires hebdomadaires des couples Processus-Parent et Processus-Enfant, en relation avec le nombre de machines et le nombre d'exécutions du couple.}
\item {Entrainement de l’algorithme de détection :}
\item {État de l'art sur les modèles de machine Learning (non supervisé et supervisé)}
\item {Support Vector Machine, linear regression, NLP, Clustering, PCA}
\item {Data Visualisation}
\item {Mise en place de métriques d'évaluations :}
\item {Quantitatives: recall, précision et F-score sur multiples datasets}
\item {Qualitative: Relecture des résultats avec les analystes de sécurité}
\item {Mise en production de l’outil :}
\item {Réduction de la quantité de calculs et de la mémoire nécessaire}
\item {Réécriture du code pour permettre un calcul distribué}
\item {POC en Reinforcement Learning:}
\item {Mise en place du modèle d'Alpha Zero adapté au Puissance-4, sur 2 semaines.}
\end{tightitemize}
}
\experiencetechno{• Splunk (Gestionnaire de l'information des événements de sécurité)
• Python / librairies de Machine Learning (Numpy, Pandas, Scikit Learn, …)
• Regular Expressions
}
\sectionspace
