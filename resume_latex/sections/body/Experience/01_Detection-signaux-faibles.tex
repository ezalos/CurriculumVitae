\experienceheader{Stage de Recherche en Machine Learning}{Natixis}{02/2020 - 08/2020}
\experiencedescription{La détection d'attaques furtives par analyse de signaux faibles

Aujourd'hui, la lutte contre les malwares dans les entreprises est principalement une défense en réponse. Dans un but de défense proactive, ce projet fut l'occasion d'étudier et de mettre en pratique les dernières avancées permises par l'IA.
À travers la méthodologie Agile, ce stage s'est déroulé lors de la première édition du lab 42 / Natixis.
}
\experiencedoublelist{La détection d'attaques furtives par analyse de signaux faibles

Aujourd'hui, la lutte contre les malwares dans les entreprises est principalement une défense en réponse. Dans un but de défense proactive, ce projet fut l'occasion d'étudier et de mettre en pratique les dernières avancées permises par l'IA.
À travers la méthodologie Agile, ce stage s'est déroulé lors de la première édition du lab 42 / Natixis.
}{La détection d'attaques furtives par analyse de signaux faibles

Aujourd'hui, la lutte contre les malwares dans les entreprises est principalement une défense en réponse. Dans un but de défense proactive, ce projet fut l'occasion d'étudier et de mettre en pratique les dernières avancées permises par l'IA.
À travers la méthodologie Agile, ce stage s'est déroulé lors de la première édition du lab 42 / Natixis.
}
\experiencetechno{• Splunk (Gestionnaire de l'information des événements de sécurité)
• Python / librairies de Machine Learning (Numpy, Pandas, Scikit Learn, …)
• Regular Expressions
}
\sectionspace
